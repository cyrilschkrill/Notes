\documentclass{article}
\usepackage{amsmath}
\usepackage{amssymb}
\newtheorem{theorem}{Théorème}
\newtheorem{definition}{Définition}
\newtheorem{proposition}{Proposition}
\newtheorem{lemme}{Lemme}
\newtheorem{corollaire}{Corollaire}
\newtheorem{remarque}{Remarque}
\newtheorem{exemple}{Exemple}
\newtheorem{exercice}{Exercice}
\title{Fonctions spéciales}
\author{cyrilschkrill}

\begin{document}
\maketitle
\begin{definition} Fonction exponentielle complexe \\
	$$ \mathrm{e}^Z = \sum_{k\geq{}0} \frac{z^k}{k!} \quad ( \forall{} z \in{} \mathbb{C}) $$ 
\end{definition}

\begin{proposition} \text{ } \\
	 La fonction exponentielle est une fonction entière de $ \mathbb{C}$ à valeurs dans $ \mathbb{C}^\star$. 
\end{proposition}

\begin{proposition} \text{ } \\
 	$\frac{\mathrm{d}}{\mathrm{d}z} \mathrm{e}^z  = \mathrm{e}^z \quad  (\forall{}z\in\mathbb{C})$ 
\end{proposition}

\begin{theorem} \text{ } \\
	(i) ( \forall{} \omega \in{} \mathbb{C}^*)( \exists{} z \in{} \mathbb{C}:\, w=\mathrm{e}^z\,) \\
\begin{corollaire}  \text{ }  \\	
\begin{center}
	\begin{tabular}{ c c c c }
		$v\;:\;(0,+\infty)$ & $\times$ & $(-\pi,\pi]$ & \longrightarrow \mathbb{C}^*$ \\
	  	    r           &,       & \theta     & \longmapsto  r\cdot{}\mathrm{e}^{i\theta}	
	\end{tabular} 
  \end{center}
est bijective et continue. \\
En revanche sa réciproque n'est pas continue en général.
\end{collolaire}



	
\end{document}
